\documentclass[xcolor={table}]{beamer}
\usepackage[utf8]{inputenc}
\usepackage{minted}
\useoutertheme{infolines}

%%%%% Title and Author %%%%%
\title{Why are VR Jets Bumpy?}
\author[dguest@cern.ch]{Dan~Guest}
\institute[UCI]{UC~Irvine}

\definecolor{UCIBlue}{RGB}{0,100,164}
\definecolor{UCIYellow}{RGB}{255,210,0}
\definecolor{UCIYDull}{RGB}{247,235,95}
\definecolor{UCIBDull}{RGB}{106,162,184}
\definecolor{UCIBDark}{RGB}{27,61,109}
\usecolortheme{crane}
\setbeamercolor*{palette primary}{use=structure,fg=white,bg=UCIBlue}
\setbeamercolor*{palette secondary}{use=structure,fg=white,bg=UCIYellow}
\setbeamercolor*{palette tertiary}{use=structure,fg=white,bg=UCIBDark}

\setbeamertemplate{enumerate items}[default]
\setbeamertemplate{navigation symbols}{}
\setbeamercovered{transparent}
\usefonttheme{serif} % default family is serif
\newcommand{\um}{\mu \mathrm{m}}
\newcommand{\mm}{\mathrm{mm}}
\newcommand{\cm}{\mathrm{cm}}
\newcommand{\gev}{\mathrm{GeV}}
\newcommand{\mev}{\mathrm{MeV}}
\newcommand{\tev}{\mathrm{TeV}}
\newcommand{\T}{\mathrm{T}}
\newcommand{\pt}{p_{\mathrm{T}}}
\newcommand{\ptr}{\pt}
\newcommand{\ptt}{\pt^{\text{s}}}
  \newcommand{\met}{E_{\mathrm{T}}^{\mathrm{miss}}}
\newcommand{\supp}[1]{\tilde{#1}}
\newcommand{\neut}{\supp{\chi}_1^0}
\newcommand{\cha}{\supp{\chi}_{1}^{\pm}}

\newcommand{\graphic}[2][0.99]{\includegraphics[width=#1\textwidth]{#2}}
\newcommand{\backupbegin}{
   \newcounter{framenumberappendix}
   \setcounter{framenumberappendix}{\value{framenumber}}
}
\newcommand{\backupend}{
   \addtocounter{framenumberappendix}{-\value{framenumber}}
   \addtocounter{framenumber}{\value{framenumberappendix}}
}
\newcommand{\link}[2]{\underline{\href{#2}{#1}}}
\newcommand{\arxiv}[1]{\link{arXiv:#1}{http://arxiv.org/abs/#1}}
\newcommand{\twocol}[3][0.5]{
  \newdimen\scwid
  \scwid=\dimexpr\textwidth-#1\textwidth\relax
  \begin{columns}
    \begin{column}{#1\textwidth}#2\end{column}
      \begin{column}{\scwid}#3\end{column}
  \end{columns}
}
\newcommand{\cent}[1]{\begin{center}#1\end{center}}

% -- set graphics path --
\graphicspath{{figures/}{code/plots/}}

% _________________________________________________________________________
% main document starts here

\begin{document}

\begin{frame}
  \maketitle
\end{frame}

\section{Smoothness}

\begin{frame}
  \frametitle{Preliminary: What is smooth?}
  \begin{itemize}
  \item Physics tells us that sometimes underlying distributions are smooth
  \item ie.
    \[ \frac{dN}{dx} = \text{something smooth} \]
    where $x$ is, say, the smooth parton $\ptt$
  \item But we can't measure $x$!
  \item Instead we measure some reconstructed thing, i.e. $\ptr$
  \item And on all our histograms, we write
    \[ \frac{dN}{d\ptr} = \text{histogram} \]
  \end{itemize}
\end{frame}

\begin{frame}
  \frametitle{Breaking it down}
  \begin{itemize}
  \item OK, so let's split up the differential
    \[ \underbrace{\frac{dN}{d\ptr}}_{\text{measured}} = \underbrace{\frac{dN}{d\ptt}}_{\text{smooth}} \underbrace{\frac{d\ptt}{d\ptr}}_{?} \]
  \item Here $\ptt$ is smooth, $\ptr$ is reconstructed jets
    %% \begin{itemize}
    %% \item It doesn't matter what $\ptt$ is, just that it can't be directly reconstructed
    %% \end{itemize}
  \item So if $\frac{d\ptt}{d\ptr}$ is smooth, we're all set. Is it?
  \end{itemize}
\end{frame}

\section{VR Jets}

\begin{frame}
  \frametitle{Jet Reconstruction: Toy model}
  \begin{itemize}
  \item Let's say some fraction $f$ of $\ptt$ gets clustered
    \[\ptr = f \cdot \ptt \]
  \item With pileup we \emph{can} have $f > 1$
  \item If the jet is not fully contained $f < 1$
  \item For our purposes, say $0 \leq f \leq 1$
    \begin{itemize}
    \item The overall scales are irrelevant here anyway
    \end{itemize}
  \item \textbf{Pileup and containment depend on the jet radius}
  \end{itemize}
\end{frame}

\begin{frame}
  \frametitle{Toy model: VR Jets}
  \framesubtitle{Fun with derivatives}
  \begin{itemize}
  \item With VR jets, radius is a function of the reco energy
  \item Factorize as $\boxed{\ptr = f(\ptr) \ptt}$
    \[ \ptr = f(\ptr) \ptt \to \ptt = \frac{\ptr}{f(\ptr)} \to d\ptt = \frac{d\ptr}{f(\ptr)} - \frac{\ptr f'(\ptr) d\ptr}{f(\ptr)^2} \]
    \[ \boxed{ \frac{d\ptt}{d\ptr} = \frac{1}{f(\ptr)} \left( 1 - f'(\ptr) \frac{\ptr}{f(\ptr)} \right) } \]
  \end{itemize}
  \begin{block}{Limiting Cases}
    \begin{itemize}
    \item $f(\ptr) = c \quad \Rightarrow \quad \frac{dN}{d\ptr} \propto \frac{1}{c}$ (squashed distribution is taller)
    \item $f(\ptr) = c \cdot \ptr$ (unstable, energy grows without bound)
    \end{itemize}
  \end{block}
\end{frame}

\begin{frame}
  \frametitle{VR Case: ``Leakage'' is Related to Area}
  \begin{itemize}
  \item In VR jets, assuming flat ``Leaky'' energy distribution
    \[ f(\ptr) = \frac{R(\ptr)^{2}}{R^{2}_{\text{max}}} = \begin{cases}
      \frac{ \rho^2 }{(\ptr R_{\text{max}})^2}  &\text{if}\quad \ptr > \frac{\rho}{R_{\text{max}}}\\
       1 &\text{if}\quad  \ptr <  \frac{\rho}{R_{\text{max}}}
    \end{cases}
      \]
    \[ f'(\ptr) = \begin{cases}
      \frac{-2 \rho^2 }{(\ptr R_{\text{max}})^3}  &\text{if}\quad \ptr > \frac{\rho}{R_{\text{max}}}\\
      0 &\text{if}\quad  \ptr <  \frac{\rho}{R_{\text{max}}}
    \end{cases}
    \]
  \item In reality, the jet is mostly contained
  \item Call this containment $c$
    \[ f(\ptr) = \overbrace{c}^{\text{contained}} + \overbrace{(1-c) \frac{R(\ptr)^{2}}{R^{2}_{\text{max}}}}^{\text{leaky}} = \begin{cases}
      \frac{ (1-c) \rho^2 }{(\ptr R_{\text{max}})^2} + c  &
      \text{if}\quad \ptr > \frac{\rho}{R_{\text{max}}} \\
      1 & \text{if}\quad  \ptr <  \frac{\rho}{R_{\text{max}}}
    \end{cases}
      \]
  \end{itemize}
\end{frame}

\section{Toy Spectra}

\begin{frame}
  \frametitle{Toy Spectrum: $\ptt \exp(\ptt / 30\, \text{GeV})$}
  \twocol{
    \begin{center}
      Rafael's (real)
      \graphic{trkjet1_pt.png}
    \end{center}
  }{
    \begin{center}
      Toy Spectrum
      \graphic{{spec.pdf}}
    \end{center}
  }
  \begin{itemize}
  \item I don't know the containment fraction, but:
    \begin{itemize}
    \item Just playing with numbers, get something close to $bb + \text{ISR}$
    \item In $bb + \text{ISR}$, the bump got worse with the ``bug''
    \item Bug added lots of pileup tracks $\to$ less contained
    \end{itemize}
  \end{itemize}
\end{frame}

\section{Conclusions}

\begin{frame}
  \frametitle{Summary}
  \begin{itemize}
  \item Main point: $f'$ has a discontinuity at $\ptr = \rho / R_{\text{max}}$
  \item Given this:
    \[ \frac{d\ptt}{d\ptr} = \frac{1}{f(\ptr)} \left( 1 - f'(\ptr) \frac{\ptr}{f(\ptr)} \right) \]
    we expect this discontinuity to put a step in the $\ptr$ spectrum
  \item The size of the bump depends heavily on the jet containment
    \begin{itemize}
    \item Rafael saw a bigger bump when pileup increased
    \end{itemize}
  \item \textbf{This might be a feature of the jet algorithm}
  \end{itemize}
\end{frame}

\begin{frame}
  \frametitle{Discussion}
  \begin{enumerate}
  \item Someone should check my math
    \begin{enumerate}[a]
    \item I made a lot of approximations
    \item In the time this took me to do, could have just run fastjet
    \end{enumerate}
  \item This doesn't say there \emph{isn't} a bug in $bb + \text{ISR}$
  \item What do we need to do about this?
    \begin{enumerate}[a]
    \item Maybe nothing: is it well modeled?
    \item For functional fits: make a smoother $R(\ptr)$
    \end{enumerate}
  \end{enumerate}
\end{frame}

% put slides

\end{document}
